\documentclass[fleqn, 11pt]{beamer}
% Class options include: notes, notesonly, handout, trans,
%                        hidesubsections, shadesubsections,
%                        inrow, blue, red, grey, brown

% Theme for beamer presentation.
\usepackage{beamerthemesplit} 
\usepackage{setspace, float, caption, subcaption, slashed}
\usepackage{amsmath, amssymb, nccmath, physics, tensor, listings, verbatim}
\usepackage{xparse, chngcntr, graphicx, hyperref, wrapfig, physunits}
\usepackage{tikz} 
\usepackage[compat=1.1.0]{tikz-feynman}
\usetikzlibrary{shapes,arrows,positioning,automata,backgrounds,calc,er,patterns}

\title{So You Wanna Get Good At \LaTeX}

\author{Marcell Howard}

\begin{document}

\begin{frame}
	\titlepage
\end{frame}

\begin{frame}
	\frametitle{Useful Packages}
	
	\begin{itemize}
		\item<1->There are a bunch of packages we can import but which ones \emph{should} we import? We'll break it up in order of decreasing importance
		\begin{itemize}
			\item<2-> amsmath, amssymb, graphicx,
			\item<3-> physics, hyperref, caption, subcaption, wrapfig, xparse, physunits
			\item<4-> inputenc, chngcntr, nccmath, tensor,
		\end{itemize}
	\end{itemize}
\end{frame}

\begin{frame}
	\frametitle{The Necessary Packages}
	
	\begin{itemize}
		\item<1-> amsmath
		\begin{itemize}
			\item<2-> Contains every environment to write equations in
			\item<2-> Allows for specified alignment of objects in aforementioned environments
		\end{itemize}
		\item<1-> amssymb
		\begin{itemize}
			\item<3-> Contains just about any math symbol you've encountered in your life
		\end{itemize}
		\item<1-> graphicx
		\begin{itemize}
			\item<4-> Use if want to include figures
		\end{itemize}
	\end{itemize}
\end{frame}

\begin{frame}
	\frametitle{These Packages Makes Things Look Nice}
	
	\begin{itemize}
		\item<1-> physics
		\begin{itemize}
			\item<2-> Contains just about everything a physicist could want
			\begin{itemize}
				\item<2-> Macros for special/transcendental functions, derivatives, matrices, etc.
			\end{itemize}
		\end{itemize}
		
		\item<1-> hyperref
		\begin{itemize}
			\item<3-> Makes the ref command hyperlinkable
			\item<4-> Makes urls hyperlinkable in bibliography
		\end{itemize}
		
		\item<1-> caption/subcaption
		\begin{itemize}
			\item<5-> Provides flexibility for captioning figures and subfigures
		\end{itemize}
		
		\item<1-> wrapfig
		\begin{itemize}
			\item<6-> Allows for figures to be wrapped around the text
		\end{itemize}
		
		\item<1-> xparse
		\begin{itemize}
			\item<7-> Allows for creating new commands (using the -newcommands\{\}\{\})
		\end{itemize}
		
		\item<1-> physunits
		\begin{itemize}
			\item<8-> Nice and intuitive commands for SI units
		\end{itemize}
	\end{itemize}
\end{frame}

\begin{frame}
	\frametitle{You Don't Need These (But They're Nice)}
	
	\begin{itemize}
		\item<1-> nccmath
		\begin{itemize}
			\item<2-> Basically the amsmath package but more
		\end{itemize}
		
		\item<1-> tensor
		\begin{itemize}
			\item<3-> Allows for phantom indices for coding matrices i.e. $\tensor{A}{^\mu_\nu}$ instead of $A^\mu_\nu$
		\end{itemize}
		
		\item<1-> chngctr
		\begin{itemize}
			\item<4-> Resets counter for equations/figures/tables etc.
		\end{itemize}
		
		\item<1-> inputenc
		\begin{itemize}
			\item<5-> Sets the unicode version for \LaTeX but is now obsolete
		\end{itemize}
	\end{itemize}
\end{frame}

\begin{frame}
	\frametitle{Useful Commands}
	
	\begin{itemize}
		\item<1-> physics
		\begin{itemize}
			\item<2-> qty
			\begin{itemize}
				\item<3-> Can't use when using newline in an equation environment. Must use -left and -right with -. at the beginning and end of each line break
			\end{itemize}
			\item<2-> vector notation
			\begin{itemize}
				\item<4-> -vb\{a\}, -va\{a\}, -vu\{a\},
			\end{itemize}
			\item<2-> special functions
			\begin{itemize}
				\item<5-> -sin, -cos, -ldots, $\rightarrow \sin, \cos, \ldots$
				\item<5-> -exp, -log, -ln, -erf, -ldots $\rightarrow \exp, \log, \ln, \erf, \ldots$
				\item<5-> -det, -Pr, -Tr, -rank, -ldots $\rightarrow \det, \Pr, \Tr, \rank, \ldots$
			\end{itemize}
		\item<2-> matrix macros
		\begin{itemize}
			\item<6-> -mqty(-imat\{3\}), -mqty(-pmat\{3\}), -mqty(-zmat\{2\}\{2\}) $\rightarrow \mqty(\imat{3}), \mqty(\pmat{3}), \mqty(\zmat{2}{2})$
		\end{itemize}
		\end{itemize}
	\end{itemize}

\end{frame}

\begin{frame}
	\frametitle{Useful Commands Cont.}
	
	\begin{itemize}
		\item<1-> amsmath
		\begin{itemize}
			\item<2-> align
			\begin{itemize}
				\item<3-> Aligns equations via placement of \&
				\item<3-> Updates equation counter
			\end{itemize}
		\item<2-> split
		\begin{itemize}
			\item<4-> Also aligns equations via placement of \&
			\item<4-> Does \emph{not} update equation counter
		\end{itemize}
		\end{itemize}
	
	\item<1-> physunits
	\begin{itemize}
		\item<5-> Putting prefixes in front of units is very convenient i.e. kilometers becomes -m[k] to get $\m[k]$ and gigayears becomes -y[G] to get $\y[G]$
	\end{itemize}
	\end{itemize}
\end{frame}

\begin{frame}
	\frametitle{More useful commands}
	
	\begin{itemize}
		\item<1-> Can use the back slash for spacing in text and equations
		\item<2-> Using \~\ for a non-breaking space (so write Marcell~Howard to prevent a line break between Marcell and Howard)
		\item<3-> Spacing in equations (-; = thick -: = medium -, = thin, - (space) = single space, -! = negative thin space,)
		\item<4-> Can use a double back slash for newlines
		\item<5-> The -rm command un-italicizes letters
	\end{itemize}
\end{frame}

\begin{frame}
	\frametitle{For Big Projects}
	
	\begin{itemize}
		\item<1-> We can store the preamble, chapters, etc. in different TeX files and call them!
		\item<2-> Use the -input, -include, or -import functions
	\end{itemize}
\end{frame}




































































































\end{document}